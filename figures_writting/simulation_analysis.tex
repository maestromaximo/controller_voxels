\documentclass[12pt]{article}
\usepackage{amsmath}
\usepackage{graphicx}
\usepackage{float}
\usepackage{booktabs}
\usepackage{geometry}

\geometry{left=1in, right=1in, top=1in, bottom=1in}

\begin{document}

\section{Numerical Simulations}

To validate the proposed Linear Quadratic Regulator (LQR) control scheme, we performed numerical simulations on a 2D slice of the proposed voxelated volume. The system model represents a $0.3\,\text{m} \times 0.3\,\text{m}$ enclosed air volume, partitioned into a $5 \times 5$ grid of voxels.

\begin{figure}[H]
    \centering
    \includegraphics[width=0.4\textwidth]{sim_outputs/voxelModel.png}
    \caption{2D 5x5 Voxel model.}
    \label{fig:model1}
\end{figure}

\subsection{Physical Parameters}
The simulation parameters were derived based on the physical properties of air at standard temperature and pressure, with the small heater/sensor nodes suspended in each voxel. The parameters strictly follow the linear model derived in Section~\ref{sec:modelling}, assuming pure conduction between nodes.

The heater nodes possess a significantly higher heat capacity than the air surrounding them ($c_{heater} \approx 1.6\,\text{J/K}$ vs $c_{air} \approx 0.26\,\text{J/K}$), leading to a system where the actuator dynamics are slower than the plant dynamics. The coupling coefficients $k_{ij}$ and $\kappa_i$ are derived from the thermal conductivity of air ($k_{th} \approx 0.026\,\text{W/mK}$).

The derived model coefficients are summarized in Table~\ref{tab:params}.

\begin{table}[h]
    \centering
    \caption{Simulation Parameters for Air Volume ($0.3\times0.3$m)}
    \label{tab:params}
    \begin{tabular}{@{}llcl@{}}
        \toprule
        Parameter & Symbol & Value & Unit \\
        \midrule
        Voxel Size & $L$ & $0.06$ & m \\
        Density (Air) & $\rho$ & $1.18$ & kg/m$^3$ \\
        Specific Heat (Air) & $c_p$ & $1005$ & J/(kg$\cdot$K) \\
        Thermal Cond. (Air) & $k_{th}$ & $0.026$ & W/(m$\cdot$K) \\
        \midrule
        Heat Capacity (Air) & $c_i$ & $0.26$ & J/K \\
        Heater Capacity & $c_i^*$ & $1.60$ & J/K \\
        Coupling Rate & $k_{ij}$ & $0.006$ & s$^{-1}$ \\
        Heater Coupling & $\kappa_i$ & $0.06$ & s$^{-1}$ \\
        Env. Loss Rate & $k^e_i$ & $0.006$ & s$^{-1}$ \\
        Max Power & $p_{max}$ & $5.0$ & W \\
        \bottomrule
    \end{tabular}
\end{table}

\subsection{Simulation 1: Single Voxel Step Response}
The first test commands the center voxel to $100^\circ$C. Figure~\ref{fig:sim1} shows the result.
With the pure conduction model, the system response is relatively slow (settling time $>10$s) due to the low thermal conductivity of air. The "leakage" to neighbors is minimal because $k_{ij}$ is very small ($0.006\,\text{s}^{-1}$). The heater quickly reaches a higher temperature to drive heat into the voxel, constrained by the local coupling $\kappa_i$.

\begin{figure}[H]
    \centering
    \includegraphics[width=0.8\textwidth]{sim_outputs/sim1_center_step.png}
    \caption{Step response of the center voxel. The system exhibits slow, overdamped dynamics characteristic of diffusion in air.}
    \label{fig:sim1}
\end{figure}

\subsection{Simulation 2: Uniform Heating with Saturation}
Figure~\ref{fig:sim2} shows the uniform heating test ($20^\circ$C to $100^\circ$C).
The inputs saturate at $5\text{W}$ initially. The temperature rises linearly while saturated, then exponentially decays to the target. The uniformity is excellent due to the symmetry of the grid and the relatively weak environmental coupling.

\begin{figure}[H]
    \centering
    \includegraphics[width=0.8\textwidth]{sim_outputs/sim2_uniform_sat.png}
    \caption{Uniform step response. Saturation at 5W limits the initial rise rate.}
    \label{fig:sim2}
\end{figure}

\subsection{Simulation 3: Gradient Maintenance}
The gradient test (Figure~\ref{fig:sim3}) demonstrates the high spatial resolution possible in this medium. Because $k_{ij}$ is small, the thermal crosstalk between the hot (left) and cold (right) zones is weak. The error map shows that the controller can maintain the $20^\circ$C target on the cold side with very little error, as the passive heat leak from the hot side is easily overwhelmed by natural environmental losses or simply results in a very slow temperature creep that the simulation horizon captures as a small steady-state offset.

\begin{figure}[H]
    \centering
    \includegraphics[width=\textwidth]{sim_outputs/sim3_gradient.png}
    \caption{Steady-state gradient maintenance. The low conductivity of air allows for sharp thermal gradients to be maintained with minimal error on the cold side.}
    \label{fig:sim3}
\end{figure}

\end{document}
